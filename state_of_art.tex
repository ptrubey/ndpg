\subsection{Current trends of extreme value theory in novelty detection}
\add{\bf This whole section should be condensed to just one paragraph, and
included in the introduction.}

The intersection of extreme value theory and anomaly detection is a current 
    topic of research.  Some methods employ EVT on densities estimated through 
    other means, such as \cite{clifton2011} using a Gaussian Mixture model, and 
    \cite{gu2021} using a Gaussian process, then both employ EVT on the 
    estimated densities to establish a decision threshold.  In this manner, they
    avoid setting the threshold heuristically.

\cite{goix2017} employs a theoretical argument from EVT to establish a threshold
    such that it partitions the space $[1,\infty)^d$ into $\alpha$-cones, where 
    $\alpha$ indicates which dimensions exceed the threshold.  It then submits 
    real data to the standard Pareto transformation,
    $T(x) = \frac{1}{1 - \hat{F}(x)}$, and bins the data according to which 
    $\alpha$-cone it falls in.  $\alpha$-cones with few observations are 
    analogous to lower-density regions, so observations falling into these cones
    are considered more likely to be anomalous.

\makenote{expand}

% EOF