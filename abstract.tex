\add{\bf{We need and abstract!}}

A fundamental result of multivariate extreme value theory is independence
    in distribution between the angular component and radial component of an
    extreme observation.   We make use of this independence to develop an 
    anomaly detection algorithm that separates the \emph{anomaly score} into 
    independent angular and radial scores.  These scores are then developed
    separately as functions of calculated density.

To fit an angular density, we project the angular component onto 
    $\mathbb{S}_p^{d-1}$ and fit a Bayesian non-parametric model implementing 
    projected gamma as the kernel density.  This works well for fitting a model, 
    but actual density estimation for use in anomaly scores is unstable.
    Instead, we use the fitted model to generate a posterior predictive sample, 
    and make use of non-parametric density estimation algorithms to generate
    anomaly scores.  We offer two such angular scores, based on the $k$-nearest
    neighbors, and kernel density estimation algorithms respectively.

We expand the applicability of our approach by implementing a categorical data
    model, making use of Dirichlet-multinomial as the kernel density.  We then 
    adapt the posterior predictive density scores developed for the angular 
    data to the categorical data model.


% EOF