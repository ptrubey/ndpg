A fundamental result of multivariate extreme value theory is independence
    in distribution between the angular component and radial component of an
    extreme observation.  We make use of this independence to develop an 
    anomaly detection algorithm that separates the \emph{anomaly score} into 
    independent angular and radial scores.  These scores are then developed
    separately as functions of calculated density.  We induce applicability of
    EVT in real data by considering the peaks-over-threshold regime,
    transforming only data in excess of a large threshold to be standard 
    multivariate Pareto.

To fit an angular density, we project the angular component onto 
    $\mathbb{S}_p^{d-1}$ and fit a Bayesian non-parametric model implementing 
    projected gamma as the kernel density. This works well for fitting a model, 
    but density estimation for use in anomaly scores is unstable.
    Instead, we use the fitted model to generate a posterior predictive sample, 
    and make use of non-parametric density estimation algorithms on that sample
    to compute anomaly scores.  We offer two such angular scores, based on the 
    $k$-nearest neighbors, and kernel density estimation algorithms 
    respectively.

We further expand the applicability of our approach by implementing a 
    categorical data model, making use of Dirichlet-multinomial as the kernel 
    density.  We then adapt the posterior predictive density scores developed 
    for the angular data model to the categorical data model.

Finally, we combine the regimes to create a \emph{mixed} data model.  We further
    adapt the scores developed in the angular and categorical data models to the
    mixed data regime.  We analyze the performance of our methods in the 
    categorical and mixed data regimes.  Additionally, we investigate the 
    standard Pareto rank-ordered transformation, which expands the applicability
    of our method to all real-valued data, not only that in excess of a 
    threshold.

We find anomaly scores developed in our methods to perform generally better than
    the canonical methods we tested.

% EOF