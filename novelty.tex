\section{Novelty Detection Methods\label{sec:novelty}}
As previously stated, a novelty detection algorithm produces an anomaly score 
    which provides a ranked ordering of which observations are likely to be 
    anomalous, with higher scores indicating more likely anomalous. Building on 
    the notion that anomalies occur in areas of low density, a general Bayesian
    anomaly score for observation $x_i$, can be defined as
    \[
    S_i = \left[\int_{\Theta}f(x_i\mid\theta)dG(\theta\mid\mathcal{D})\right]^{-1}
    \]
    where $\mathcal{D}$ is the observed data and $\theta$ the distributional
    parameters.  That is, the reciprocal of the posterior predictive density 
    at observation $x_i$.

Given the independence between the angular and radial components of an
    extreme observation, we can consider sub-scores for the radial 
    and angular components independently.  That is,
    \[
        S_i = S_{i,r}\times S_{i,\bm{v}} = 
            f_r(r_i)^{-1}\times
            \left[\int_{\Omega} f_v(\bm{v}_i\mid\bm{\alpha})
                \,\text{d}G(\bm{\alpha}\mid\mathcal{D})\right]^{-1}
    \]
    By construction $r_i$ follows a standard Pareto distribution, so its
    density is $f_r(r_i) = r_i^{-2}$.  As previously discussed in
    Section~\ref{sec:evt}, the kernel $\mathcal{PG}_\infty$, needed for 
    density estimation on the surface of ${\mathbb S}_{\infty}^{d-1}$ is 
    not available in analytic form, thus, we resorted to transforming the
    data to $\mathbb{S}_p^{d-1}$ for a large but finite $p$.
    This makes estimation of distributional parameters possible, 
    but in the context of anomaly detection, a score based on
    $\mathcal{PG}_p$ is problematic.  Recall in 
    Equation~(\ref{eqn:projgamma}), the value of the Jacobian is dependent
    upon which dimension is replaced in the transformation.  This implies that
    under uniform $\bm{\alpha}$, calculated density can be changed by
    permuting the order of dimensions.  This is problematic in an anomaly
    detection application, because a relative ordering of density between 
    observations is specifically what we're trying to calculate.
    % \add{\bf Can you make this argument more clear? I am not sure I get it.}
    % \makenote{I think its now sufficiently clear?}
    Further, for an observation near 0 evaluated on a gamma-related distribution 
    with a shape parameter $\alpha$ near 0, evaluated density can be divergent.
    This instability would play havoc on any directly 
    calculated density-based scores.

To avoid the problems of angular density evaluation we use a 
    non-parametric angular density estimator based on a sample from the 
    posterior predictive distribution of the model developed in 
    Section~\ref{sec:evt}. Here, we consider two well-established
    methods: $k$-nearest neighbors, or \emph{$k$NN} \citep{mack1979}, and 
    kernel density estimation, or \emph{KDE} \citep{parzen1962}.  For both of 
    these methods we make use of pairwise distances between observations 
    from the dataset, and replicates from a posterior predictive sample.  

As described in \cite{trubey:pg}, the  geodesic distance on 
    $\mathbb{S}_{\infty}^{d-1}$ 
    is expensive to evaluate, as the computational burden grows 
    combinatorically with the number of dimensions.  As an alternative they propose an 
    estimate of distance that is computationally cheap to evaluate, bearing 
    a cost equivalent to that of a Euclidean norm.
    % \makenote{option 1}
    % For observations on different faces of 
    % $\mathbb{S}_{\infty}^{d-1}$, a performant analogue to distance can be
    % evaluated by rotating the face of the second observation along the axis of
    % the first observation, and computing the Euclidean norm between the first
    % observation and the transformed second observation.  
    % \makenote{option 2}
    Let
    ${\mathbb C}_{\ell}^{d-1} = \lbrace \bm{x} : 
        \bm{x} \in {\mathbb S}_{\infty}^{d-1}, x_{\ell} = 1\rbrace$
    comprise the $\ell$th \emph{face} of $\mathbb{S}_{\infty}^{d-1}$.  For a 
    pair of points on the same face, the Euclidean distance corresponds to the
    geodesic, or length of the shortest possible path between those two points.  
    For a pair of points 
    $\bm{a} \in \mathbb{C}_{\ell}^{d-1},\;\bm{b}\in\mathbb{C}_{\jmath}^{d-1}$,
    we can rotate $\mathbb{C}_{\jmath}^{d-1}$ into the same hyperplane as 
    $\mathbb{C}_{\ell}^{d-1}$.  Transform $\bm{b}$ such that %\\
    \begin{equation}
        \label{eqn:rotation}
        \bm{b}^{\prime} = P_{\jmath\ell}(\bm{b}) = 
        \begin{cases}
            b_{i} &\text{for }i\neq \jmath,\ell\\
            1 &\text{for }i = \ell\\
            2 - b_{\ell} &\text{for }i = \jmath
        \end{cases}\;\hspace{2cm}\;
        g(\bm{a},\bm{b}) = \lVert \bm{a} - \bm{b}^{\prime}\rVert_2
    \end{equation}
    After transformation, the Euclidean norm between $\bm{a}$ and 
    $\bm{b}^{\prime}$ corresponds to a negative definite kernel that
    provides an upper bound on geodesic distance on 
    $\mathbb{S}_{\infty}^{d-1}$ between $\bm{a}$ and $\bm{b}$.

% \add{\bf This whole section needs to rewritten. First we have to connect to 
%     the previous section. As a result of what the methods presented in the 
%     previous section we have a sample from the angular density that 
%     characterizes the extreme behavior of our observations. We have samples, 
%     but we have no way to evaluate it. At this point you provide information 
%     about estimated of the density, and then you have to mention that a 
%     measure of distance on ${\mathbb S}_{\infty}^{d-1}$, and you refer to the 
%     previous paper.}

\subsection{$k$-Nearest Neighbors Density estimation}
We use the kernel $g$ defined in Equation \ref{eqn:rotation} to obtain 
    a local posterior predictive density based on a $k$NN estimator 
    on $\mathbb{S}_{\infty}^{d-1}$. To this end we consider a locally 
    uniform density within a $d-1$--dimensional ball $\mathbb{B}$, centered on 
    observation $\bm{v}_i$.  The radius $D_{k}(\bm{v}_i)$ is calculated as 
    $g(\bm{v}_i, \bm{v}_{N(i)})$, where $\bm{v}_{N(i)}$ is the $k$th nearest 
    neighbor of $\bm{v}_i$ in a sample from the 
    posterior predictive distribution. The volume of the ball is calculated as
    \begin{equation}
        \label{eqn:vol_sphere}
        \text{Vol}(\mathbb{B}_k^{d-1}) =
        \frac{\pi^{\frac{d-1}{2}}D_{k}(\bm{v}_i)^{d-1}}{
            \Gamma\left(\frac{d-1}{2} + 1\right)}.
    \end{equation}
    The density is thus estimated as 
    $f_{\bm{v}}^{(k\text{NN})}(\bm{v}_i\mid \bm{V}) \approx 
        \frac{k}{N}\left(\text{Vol}(\mathbb{B}_{k}^{d-1})\right)^{-1}$
    where $N$ is the total number of replicates of from the posterior predictive
    distribution.  Taking the reciprocal of the estimated angular density, 
    the angular score under the $k$NN estimator is then
    \begin{equation}
        \label{eqn:ad_knn_h}
        S_{i,\bm{v}}^{\text{$k$nn}} = \frac{N}{k}
            \frac{\pi^{\frac{d-1}{2}}D_{k}(\bm{v}_i)^{d-1}}{
            \Gamma\left(\frac{d-1}{2} - 1\right)}
    \end{equation}
    In our experience, using a large posterior predictive sample, 
        \makenote{this should be quantified}
    the resulting ordering of scores was relatively robust to a choice of $k$ 
    between 2 and 10.  We used $k = 5$ in our performance analysis.
% When implementing this analysis on the hypercube, we lose one degree of freedom,
%     and as a stand-in for distance, we use the negative definite kernel metric 
%     previously explored in~\cite{trubey:pg} as a means of establishing a proper 
%     scoring rule on $\mathbb{S}_{\infty}^{d-1}$.  Thus, the score becomes
%     \begin{equation}
%         \label{eq:ad_knn_h}
%         S_i^{hKNN} = \frac{r_i^{2}N}{k}
%         \frac{\pi^{\frac{d-1}{2}}D_{k}(\bm{v}_i)^{d-1}}{
%           \Gamma\left(\frac{d-1}{2} - 1\right)}
%   \end{equation}
%     This is of course an approximation to the true posterior predictive density 
%     at observation $i$.  In addition to the noise induced by using random 
%     samples from the posterior predictive, there is some bias induced in the 
%     calculated volume
%     if $D_k(\bm{v}_i)^{d-1}$ is greater than the distance from $\bm{v}_i$ to 
%   unity ($\bm{1}_d$), or from $\bm{v}_i$ to some axis.  In practice,
%   this induced bias seems to have relatively small effect. 
%   \makenote{worth looking into what proportion of observations experience this 
%   bias, and what performance outcome is}.
%   In our experience, using a large posterior predictive sample, the resulting 
%   ordering of scores in $k$NN based methods was robust to the choice of $k$, for 
%   values between 2 and 10.

\subsection{Kernel Density Estimation}
Kernel density estimation is an approach that makes use of 
    kernel smoothing to produce a semi-parametric estimate of the density
    function for a dataset.  For a scalar bandwidth parameter $h$,
    \[
        f_n(x) \;=\; 
            \int_{\Omega}\frac{1}{h}
                K\left(\frac{ \bm{x} - \bm{s}}{h}\right)dF_n(s) \;\approx\;
            \frac{1}{nh}\sum_{j = 1}^n
                K\left(\frac{ \bm{x} - \bm{x}_j^{*}}{h}\right)
    \]
    where $\bm{x}_j^{*}$ are random replicates from $F$.  The choice of kernel function 
    $K$, and selection of the bandwidth parameter $h$ are both topics that have
    been extensively researched. In practice the Gaussian kernel seems to be 
    well regarded for its simplicity, flexibility, and interpretability.  The
    bandwidth parameter in this case corresponds to the standard deviation 
    of the kernel function.  The multivariate Gaussian kernel is more flexible,
    accepting a matrix as the bandwidth parameter.  A larger bandwidth serves 
    to smooth the resulting density estimate, where a lower bandwidth is more 
    responsive to individual observations of data.  Optimization of $h$ is 
    application and data specific, but there do exist various 
    \emph{rules of thumb} based on summary statistics of the data. For our 
    analysis, we are making use of a distance analogue on 
    $\mathbb{S}_{\infty}^{d-1}$ described in Equation~(\ref{eqn:rotation}), 
    which precludes the ability to describe bandwidth using a matrix.  We 
    therefore consider the univariate case of $f$ in kernel space, where 
    $\lVert x - x^*\rVert$ has been replaced with $g(\bm{v}, \bm{v}^*)$.

For selection of the bandwidth parameter $h$, we employ Silverman's rule of
    thumb \citep{silverman2018}, estimating 
    $\hat{h} = \left(\frac{4}{d+2}\right)^{\frac{1}{d+4}}
        n^{-\frac{1}{d+4}}\hat{\sigma}$.
    This then requires the estimation of $\hat{\sigma}$, which in this case we
    calculate from pairwise distances.  Recall that for a random variable $X$,
    $\text{E}\left[\left\lVert X_j - X_k\right\rVert_2\right] 
        = 2\text{Var}(X)$.
    In that case, $\hat{\sigma} = 
        \sqrt{\frac{1}{2N(N-1)}\sum_{j\neq k}g(\bm{v}_j^*,\bm{v}_k^*)}$, where
    $\bm{v}_j^*,\bm{v}_k^*$ are replicates from the posterior predictive distribution.
    Then $\hat{\sigma}$ is used in the aforementioned rule of thumb for $h$.
    Finally, the angular score under KDE is then calculated as
    \begin{equation}
    \label{eqn:ad_kde_h}
    S_{i,\bm{v}}^{\text{kde}} = \text{E}_{\bm{v}^*}\left[\exp\left\lbrace -
    \left(\frac{g(\bm{v}_i,\bm{v}^*)}{\hat{h}}\right)^2\right\rbrace\mid\bm{v}\right]^{-1}
    \approx
    \left[\frac{1}{K}\sum_{k = 1}^{K}
    \exp\left\lbrace-\left(\frac{g(\bm{v}_i,\bm{v}_k^*}{\hat{h}}\right)^2\right\rbrace\right]^{-1}
    \end{equation}
    where $\bm{v}_k^*$ are again replicates from the posterior predictive distribution.
    We investigated other methods of calculating bandwidth, as well as searched
    the neighborhood around our bandwidth estimate for example datasets.
    The estimator following Silverman's rule of thumb as described consistently 
    produced the most performant rank ordering of angular anomaly scores on 
    tested datasets.

\begin{algorithm}[htb]
    \caption{Workflow for anomaly detection on $\mathbb{S}_{\infty}^{d-1}$.}\label{alg:adreal}
    % \begin{enumerate}[nolistsep]
        % \item
    \begin{algorithmic}[1]
        \State Take $r_i$, $\bm{y}_i$ according to Algorithm~(\ref{alg:workflow})
        \State Fit $\bm{y}\sim\mathcal{PYPG}$ from Equation~(\ref{eqn:modelsphere})
        \State From $\bm{\alpha}\mid\bm{y}$, sample
            $\bm{\varrho}_k^{*}\mid\bm{\alpha}\sim\prod_{\ell}\mathcal{G}(\alpha_{\ell})$ for $k = 1,\ldots,K$
        \State Take $\bm{v}_k^{*} = T_{\infty}(\bm{\varrho}_k^{*})$
        \State Take $S_{i,v}$ as per Equations~(\ref{eqn:ad_knn_h},\ref{eqn:ad_kde_h})
    \end{algorithmic}
\end{algorithm}

% EOF
