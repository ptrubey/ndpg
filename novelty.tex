\section{Novelty Detection Methods}
We can consider a novelty detection algorithm as an algorithm that assigns a
  novelty score $S_i$ to observation $i$.  We interpret a larger $S_i$ 
  as indicative that the observation is more likely to be anomalous.

Building on the notion that anomalies occur in areas of low density, a simple
  score might establish
  \[
      S_i = f_v(v_i\mid \bm{v})^{-1}.  
  \]
  That is, the reciprocal of the fitted density, $f_v\mid \bm{v}$ at the point $v_i$.
  As previously established in~\cite{trubey:pg}, density estimation on the 
  surface of ${\mathbb S}_{\infty}^{d-1}$ is not an easy problem, so for 
  purposes of establishing inference on the dependence structure of the data 
  they projected from ${\mathbb S}_{\infty}^{d-1}$ onto ${\mathbb S}_p^{d-1}$ for 
  some large but finite $p$.  This makes density estimation \emph{possible}, but 
  for purposes of anomaly detection, the projection onto ${\mathbb S}_p^{d-1}$ is 
  strongly dependent on \emph{which} dimension was used as a summary--that is, 
  which dimension became a function of the others, hereafter referred to as the 
  \emph{dependent} variable. In~\cite{trubey:pg}, 
  $y_d = (1 - \sum_{\ell = 1}^{d-1}y_{\ell}^p)^{1-p}$ was the dependent variable.
  This works well for establishing the projected Gamma density, and via 
  proportionality, inference on the projected Gamma parameters was not affected
  by the choice of the summary variable. However, when we look to establish 
  density estimation routines for purposes of anomaly detection, where we do 
  take into account the full calculated density, the choice of summary variable 
  does affect the calculated density.  Moreover, density on $\mathbb{S}_p^{d-1}$
  is only a proxy to density on $\mathbb{S}_{\infty}^{d-1}$, which is our goal.

One way around this issue would be the use of a non-parametric density estimation
  routine.  We consider two such well-established routines: $k$-nearest neighbors, 
  or \emph{$k$NN}, and kernel density estimation, or \emph{KDE}.

\subsection{$k$-Nearest Neighbors Density estimation}
\makenote{Needs citation for KNN}
We establish a local posterior predictive density under the $k$NN estimator by 
  assuming a locally uniform density within a ball $\mathbb{B}$, centered on 
  observation $i$, with radius $D_{k}^d(x_i)$ calculated as the distance to the 
  $k$th nearest neighbor to observation $i$ in a sample from the posterior 
  predictive distribution. The volume is then calculated as
  \begin{equation*}
    \begin{equation}
\label{eq:vol_sphere}
\text{Vol}(\mathbb{B}_d^k) = \frac{2\pi^{\frac{n}{2}}}{d}\frac{\pi^{\frac{d}{2}}D_{k}(\bm{x}_i)^d}{\Gamma\left(\frac{d}{2} - 1\right)}
\end{equation}
% EOF
  \end{equation*}
  where $D_k(\bm{x}_i)$ is the distance from $\bm{x}_i$ to the $k$th nearest 
  replicate of $\bm{X}\mid\bm{x}$.  A density is established as
  \begin{equation*}
    %\label{eq:ad_knn}
    f_{x}^{(k\text{NN})}(x_i\mid X) \approx 
          \frac{k}{N}\left(\text{Vol}(\mathbb{B}_d^k\right)^{-1}
  \end{equation*}
  where $N$ is the total number of replicates of $\bm{X}\mid\bm{x}$.  As mentioned,
  we desire a score interpreted as higher is more likely anomalous, so we take
  the score as the reciprocal of the density.

When implenting this analysis on the hypercube, we lose one degree of freedom,
  and as a stand-in for distance, we use the negative definite kernel metric 
  previously explored in~\cite{trubey:pg} as a means of establishing a proper 
  scoring rule on $\mathbb{S}_{\infty}^{d-1}$.  Thus, the score becomes
  \begin{equation}
    \label{eq:ad_knn_h}
    S_i^{hKNN} = \frac{2N}{(d-1)k}\frac{\pi^{\frac{d-1}{2}}D_{k}(\bm{v}_i)^{d-1}}{\Gamma\left(\frac{d-1}{2} - 1\right)}

% EOF.
  \end{equation}
  We can also implement a $k$NN based method on the latent cartesian
  representation, prior to projection onto the hypercube.  That is, take replicates
  of $r_i\mid \bm{v},\bm{v}_i$.  The score then becomes
  \begin{equation}
    \label{eq:ad_knn_e}
    \begin{equation}
\label{eq:ad_knn_e}
S_i^{eKNN} = \text{E}\left[\frac{2N}{dk}\frac{\pi^{\frac{d}{2}}D_{k}(R\bm{v}_i)^{d-1}}{\Gamma\left(\frac{d}{2} - 1\right)}\right]
\end{equation}
% EOF,
  \end{equation}
  where $D_k(R\bm{v}_i)$ signifies the euclidean distance to the $k$th nearest
  replicate of $R\bm{V}\mid\bm{v}$.  In this latent representation, we still have
  $d$ degrees of freedom. In our experience, the resulting ordering of
  scores in $k$NN based methods was relatively robust to the choice of $k$, for 
  values between 2 and 10.

\subsection{Kernel Density Estimation}
\makenote{needs citation for KDE}

Kernel density estimation

If we assume a gaussian kernel, then the KDE density for a generic $\bm{x}$ can 
  be approximated as
  \begin{equation*}
    \begin{equation}
    \label{eq:ad_kde}
    f_{\bm{x}}^{(\text{KDE})}(\bm{x}_i) \approx \frac{1}{N}\sum_{j = 1}^{N}\left[\frac{1}{h\sqrt{2\pi}}\exp\left\lbrace-\frac{1}{2}\left(\frac{D(\bm{x}_i,\bm{X}_j)}{h}\right)^2\right\rbrace\right]
\end{equation}
% EOF,
  \end{equation*}
  where $\bm{X}_j$ is the $j$th replicate from $\bm{X}\mid\bm{x}$.  The bandwidth
  parameter $h$ 

\begin{equation*}
  S_i^{\text{hKDE}} = \frac{1}{N}\sum_{j = 1}^{N}[h\sqrt{2\pi}\exp\left\lbrace\frac{1}{2}\left(\frac{g(\bm{v}_i,\bm{V}_j)}{h}\right)^2\right\rbrace\right]

% EOF
\end{equation*}

\begin{equation*}
  S_i^{\text{eKDE}} = \text{E}[h\sqrt{2\pi}\exp\left\lbrace\frac{1}{2}\left(\frac{\lVert R_i\bm{v}_i,R\bm{V}_j\rVert}{h}\right)^2\right\rbrace\right]

% EOF
\end{equation*}

\subsection{Incorporating radial information in novelty scores}
Finally, the methods we have established thus far establish novelty scores in 
  the \emph{angular} sense.  In the manner established in \cite{goix2017}, we 
  combine the angular scores thus far developed with the radial values 
  $r_i = \lVert Z \rVert_{\infty}$.  Recall that $r_i$ follow the standard 
  Pareto distribution, and under that distribution, $P(R > r) = r^{-1}$.  Then
  \[
    S_i^{\prime} = r_iS_i
  \]

% EOF
