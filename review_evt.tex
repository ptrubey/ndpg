\section{Review of Extreme Value Theory}
If there exists a limiting distribution on the maximum of a sample, then that limiting 
    distribution will exist within the generalized extreme value family.  In that case, 
    then observations that exceed a sufficiently large threshold will can be modelled using
    a generalized Pareto distribution. Such an effort is called \emph{peak over threshold} 
    modelling, and typically involves first establishing a sufficiently large threshold, 
    and fitting a generalized Pareto distribution to the excesses.  This action is performed
    marginally, then using the fitted parameters we transform the data to a common marginal 
    distribution so we can further analyze the dependence structure of the data.  Following 
    the generative form of the multivariate Pareto established in~\cite{ferreira2014},
    let the threshold for dimension $\ell$ be $b_{\ell}$.  Then using the generalized Pareto 
    distribution, we fit the scale parameter $a_{\ell}$, and extremal index $\chi_{e\ell}$ 
    to the excesses.  
    Using these fitted distribution parameters, we standardize the data using
    \[
        Z_{\ell} = \left(1 + \chi_{\ell}\frac{X_{\ell} - b_{\ell}}{a_{\ell}}\right)_{+}^{-\frac{1}{\chi_{\ell}}}.
    \]
    Then $\lVert \bm{Z}\rVert_{\infty}\mid\lVert \bm{Z}\rVert_{\infty} \geq 1$, will follow
    a standard Pareto distribution, and $\bm{Z}\mid \lVert \bm{Z}\rVert_{\infty} \geq 1$ 
    will follow a multivariate standard Pareto distribution.

\subsection{Separation of radial and angular components}
Assuming the existence of a limiting measure $\mu$ such that
\[
    \lim_{n\to\infty} n\text{Pr}\left[\frac{1}{n}\bm{Z} \not< \bm{z}\right] = \mu\left([\bm{0},\bm{z}]^c\right),
\]
then $\mu$ describes the asymptotic distribution of $\bm{Z}$ under extreme regions, and is 
    referred to as the \emph{exponent measure}.  Under such a condition, $\mu$ features 
    the homogeneity property $\mu(tA) = \frac{1}{t}\mu(A)$.  Now, define 
    $R = \lVert \bm{Z}\rVert_{\infty}$, and $\bm{V} = \frac{\bm{Z}}{\lVert Z\rVert_{\infty}}$
    such that $\bm{Z} = R\bm{V}$.  Then by the homogeneity property, $\mu$ can be factorized as
    \[
        \mu\left(\{\bm{Z} : \bm{V} \in A, R > r\}\right) = r^{-1}\Phi(A),    
    \]
    where $\Phi$ denotes the \emph{angular} or \emph{spectral} measure.  By this, we see 
    that in extreme regions, the distribution of the angular component of $\bm{Z}$ is 
    independent from that of the radial component $r$. This property permits us to model 
    them separately.

\subsection{Modeling the angular component}
As the radial component $R \in {\mathbb R}_+$ will follow a standard Pareto 
    distribution, the task becomes describing the distribution of the angular component 
    $\bm{V}\in {\mathbb S}_{\infty}^{d-1}$.  A distribution in this space can be
    approximated by projecting a distribution in $\mathbb{R}_+^d$ onto $\mathbb{S}_{p}^{d-1}$.
    Recall the $\mathcal{L}_p$ norm is 
    \[
        \lVert \bm{S} \rVert_p = \left(\sum_{\ell = 1}^d S_{\ell}^p\right)^{\frac{1}{p}}.
    \]
    Then letting $R = \lVert \bm{S}\rVert_p$, we can let 
    $Y_d = \left(1 - \sum_{\ell = 1}^{d-1}Y_{\ell}^{p}\right)^{\frac{1}{p}}$ then $\bm{Y} = \frac{\bm{S}}{R}\Rightarrow \bm{S} = R\bm{Y}$, 
    and the determinant of the Jacobian of the transformation becomes
    \[
        \lvert J \rvert = r^{d-1}\left[y_d^p + \sum_{\ell = 1}^{d-1}y_{\ell}^p\left(y_d^p\right)^{\frac{1}{p} - 1} \right].
    \]
    Note that the transformation is not differentiable at $p = \infty$, so we can not use this 
    transformation to directly model on ${\mathbb S}_{\infty}^{d-1}$.
    However, for a sufficiently large $p$, the space ${\mathbb S}_p^{d-1}$ will tend
    to approach $\mathbb{S}_{\infty}^{d-1}$, meaning a distribution in the latter can be 
    approximated by one in the former.  In~\cite{trubey:pg}, this projection is explored, and
    a \emph{projected gamma} distribution is developed that projects a vector of independent gamma r.v.'s 
    onto $\mathbb{S}_{p}^{d-1}$ for an arbitrary $p$.  Further, the projected gamma distribution is
    employed as the kernel distribution in a Dirichlet process mixture model.  For $\bm{y}_i\in \mathbb{S}_{p}^{d-1}$,
    \[
        \bm{y}_i \sim \mathcal{PG}_{p}(\bm{y}_i\mid\bm{\theta}_i),\hspace{0.5cm}\bm{\theta}_i\sim G,\hspace{0.5cm}G\sim \mathcal{DP}(\eta, G_0).
    \]
    In this paper, we employ this DP mixture model as the basis of our analysis.



