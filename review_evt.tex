\section{Review of Extreme Value Theory}

If there exists a limiting distribution on the maximum of a sample, then that limiting 
    distribution will exist within the generalized extreme value family.  If that is the 
    case, then the distribution of excesses above a threshold will exist within the 
    generalized Pareto family.  Modelling such excesses is called \emph{peak over threshold}
    modelling, and typically involves first establishing a sufficiently large threshold, 
    and fitting a generalized Pareto distribution to the excesses.  When working with multivariate
    data, this action is done marginally.  Then using the fitted distribution parameters,
    we transform the data to a common distribution to estimate the dependence structure.

\subsection{Separation of radial and angular components}
Assuming the existence of a limiting measure $\mu$ such that
\[
    \lim_{n\to\infty} n\text{Pr}\left[\frac{1}{n}\bm{Z} \geq \bm{z}\right] = \mu\left([\bm{0},\bm{z}]^c\right),
\]
then $\mu$ is the asymptotic distribution of $\bm{Z}$ under extreme regions, and is referred to as the 
    \emph{exponent measure}.  Under such condition, $\mu$ features the homogeneity property 
    $\mu(tA) = \frac{1}{t}\mu(A)$.  Now, define $R = \max_{\ell} Z_{\ell}$, and 
    $\bm{V} = \frac{\bm{Z}}{\max_{\ell}Z_{\ell}}$ such that $\bm{Z} = R\bm{V}$.  Then by the homogeneity 
    property, $\mu$ can be factorized as
    \[
        \mu\left(\{\bm{Z} : \bm{V} \in A, R > r\}\right) = r^{-1}\Phi(A),    
    \]
    where $\Phi$ denotes the \emph{angular} or \emph{spectral} measure.  By this, we see that in extreme 
    regions, the distribution of the angular component of $\bm{Z}$ is independent from that of the radial.  
    This property permits us to model them separately.

\subsection{Modeling the angular component}
We find a generative form for a multivariate standard Pareto model as a finite case of the 
    standard Pareto process model described in~\cite{ferreria2014}.  In this model, let 
    $Z_{\ell}\mid Z_{\ell} > 1$ marginally follow a standard Pareto distribution, and let 
    $\max_{\ell}Z_{\ell} >= 1$ for some $\ell$.  Then, following the transformation 
    $\bm{Z} = R\bm{V}$, we observe that $R = \max_{\ell}Z_{\ell}$ follows a standard
    Pareto distribution, and $\bm{V}$ follows an unspecified distribution with support on 
    ${\mathbb S}_{\infty}^{d-1}$; the positive orthant of the unit hypersphere defined under the 
    $\mathcal{L}_{\infty}$ norm.  As the dependence structure for $\bm{Z}$ is entirely
    within the distribution of $\bm{V}$, it follows that describing the distibution of $\bm{V}$ 
    effectively describes the dependence structure of $\bm{Z}$.

As in \cite{trubeypg}, we begin our model by establishing 
    \[
        Z_{\ell} = \left(1 + \chi_{\ell}\frac{X_{\ell} - b_{\ell}}{a_{\ell}}\right)_{+}^{-\frac{1}{\chi_{\ell}}}
    \]
    Then $R = \max_{\ell}Z_{\ell}$, and $\bm{V} = \frac{\bm{Z}}{R}$.

    







