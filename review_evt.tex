\section{Review of Extreme Value Theory}


\subsection{Separation of radial and angular components}
Assuming the existence of a limiting measure $\mu$ such that
\[
    \lim_{n\to\infty} n\text{Pr}\left[\frac{1}{n}\bm{Z} \geq \bm{z}\right] = \mu\left([\bm{0},\bm{z}]^c\right),
\]
then $\mu$ is the asymptotic distribution of $\bm{Z}$ under extreme regions, and is referred to as the \emph{exponent measure}.  Under such condition, $\mu$ features the homogeneity property $\mu(tA) = \frac{1}{t}\mu(A)$.  Now, define $R = \max_{\ell} Z_{\ell}$, and $\bm{V} = \frac{\bm{Z}}{\max_{\ell}Z_{\ell}}$ such that $\bm{Z} = R\bm{V}$.  Then by the homogeneity property, $\mu$ can be factorized as
\begin{equation}
    \mu\left(\{\bm{Z} : \bm{V} \in A, R > r\}\right) = r^{-1}\Phi(A),
\end{equation}
where $\Phi$ denotes the \emph{angular} or \emph{spectral} measure.  By this, we see that in extreme regions, the distribution of the angular component of $\bm{Z}$ is independent from that of the radial.  This property permits us to model them separately.

\subsection{Modeling the angular component}








