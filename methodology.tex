
\section{Binary and Categorical Data}

Many applications for which novelty detection methods might be applied will feature both real and categorical data.  \makenote{rewrite}
The most apparent shortcoming of the projected Gamma model developed in~\cite{trubey:pg} is
its inability to handle binary or categorical data.  To ameliorate this shortcoming, we propose
the following.

Let $C$ be a random categorical variable, with $K \geq 2$ categories.  Note that a binary variable is categorical, with $K = 2$. We re-code $C$ into the random vector $\bm{W}$ with length $K$, taking the form
\[
    W_k = \begin{cases}
    1 &\text{if }C = k\\
    0 &\text{otherwise}
    \end{cases} \text{ for }k = 1,\ldots, K.
\]
This coding corresponds to a multinomial random variable with size 1. To incorporate this information within our existing projected gamma structure, let us consider the Dirichlet-multinomial model.  This model is established by assuming the categorical variable distributed multinomial with a latent parameter $\pi$, and a Dirichlet prior on $\pi$. The Dirichlet-multinomial is then established by integrating out the latent parameter $\pi$. 
\[
    \bm{w} \sim \int_{\pi} \text{Multinom}(\bm{w}\mid\pi)\text{Dir}(\pi\mid\theta)\text{d}\pi
\]
As has been established in~\cite{trubey:pg}, the Dirichlet distribution is a special case of the projected gamma class, projected on the $\mathcal{L}_1$ norm.  The more general $\mathcal{PG}_1$ similarly serves as a conjugate prior for the multinomial, with posterior inference identical.  We can then write a model for categorical data as
\begin{equation}
    \label{model:cat}
    \bm{w} \sim \int_{\theta}\int_{\pi}\text{Multinom}(\bm{w}\mid \pi)\mathcal{PG}_1(\pi\mid \theta)d\pi dG(\theta).
\end{equation}
When we consider $\theta = (\alpha,\beta)$, then the latent $\pi\mid \bm{w},\theta$ can be sampled from
\begin{equation}
    \label{eqn:pifc}
    \pi\mid \bm{w},\bm{\alpha},\bm{\beta} \sim \mathcal{PG}_1(\pi\mid\alpha + \bm{w}, \beta)
\end{equation}


\section{Mixed Models}
We make a strong assumption at this point, for establishment of the angular measure,
that conditional on $\theta$, the distribution of the categorical variables is independent 
of that of the real variables.  That is, $Y \in {\mathbb S}_{p}^{d-1}$ is established
independent of $\bm{W}$.  This is not unreasonable, as the generalized Pareto parameters 
estimated to transform $\bm{X}$ into $\bm{Z}$ are estimated marginally, and the transformation $\bm{Z}\to\bm{Y}$ is independent of $\bm{W}$.  Thus we can establish a \emph{mixed} model as
\begin{equation}
    \label{model:mixed}
    (\bm{y},\bm{w})\sim \int_{\theta}\mathcal{PG}_{p}(\bm{y}\mid\theta_1)\prod_{m = 1}^M\int_{\pi_m}\left[\text{Multinom}(\bm{w}_m\mid\pi_m)\mathcal{PG}_1(\pi_m\mid\theta_{2m})d\pi_{m}\right]dG(\theta),
\end{equation}
where $m$ indexes over categorical variables in the data. By this assumption, the dependence 
structure (and associated inference) between $\bm{y}$ and $\bm{w}$ is transferred up to the 
distribution of $\theta$.
\begin{equation*}
    \label{model:mixeddp}
    \begin{aligned}
    w_i\mid \pi_i &\sim \text{Multinom}(w_i\mid \pi_i)\\
    y_i\mid\theta &\sim\mathcal{PG}_p(y_i\mid\theta^{(1)})\\
    \pi_i\mid\theta &\sim \mathcal{PG}_1(\pi_i\mid\theta^{(2)})\\
    \theta &\sim \text{DP}(\theta\mid\alpha, G_0).
    \end{aligned}
\end{equation*}

In the general projected gamma case, the integration for $\pi$ in Equation~\ref{model:cat} is not
  available in closed form.  Under a sample based inference approach, we can augment the data by 
  sampling $\pi$ according to Eqn.~\ref{eqn:pifc}, then calculate the likelihood of $\pi$ under 
  the projected Gamma distribution.  That is, for a given sample iteration,
\begin{equation*}
    \begin{aligned}
    P(\delta_i = j\mid \ldots) \propto \begin{cases} 
    n_j^{\neg i}\mathcal{PG}_p(y_i\mid\theta_j^{(1)})\mathcal{PG}_1(\pi_i\mid\theta_j^{(2)}) \hspace{0.5cm}&\text{for }j = 1,\ldots,J,\\
    \frac{\eta}{m}\mathcal{PG}_p(y_i\mid\theta_j^{(1)})\mathcal{PG}_1(\pi_i\mid\theta_j^{(2)}) \hspace{0.5cm}&\text{for }j = J + 1,\ldots, J + m.\\
    \end{cases}
    \end{aligned}
\end{equation*}
As both are projected Gamma distributions, the hyper-prior structures mentioned 
  in~\cite{trubey:pg} will still be appropriate.

\section{Novelty Detection Methods}

Building on the notion that anomalies occur in areas of low density, we consider methods of
  estimating the statistical density of observed data.  As we saw in~\cite{trubey:pg}, density
  estimation on the surface of ${\mathbb S}_{\infty}^{d-1}$ is not an easy problem, so for purposes
  of establishing inference on the dependence structure of the data we opted to project from
  ${\mathbb S}_{\infty}^{d-1}$ onto ${\mathbb S}_p^{d-1}$ for some large but finite $p$.  This makes
  density estimation \emph{possible}, but for purposes of anomaly detection, the projection onto
  ${\mathbb S}_p^{d-1}$ is strongly dependent on \emph{which} dimension was used--that is, which
  dimension became a function of the others, hereafter referred to as the \emph{dependent} variable.
  In~\cite{trubey:pg}, we declared $y_d = (1 - \sum_{\ell = 1}^{d-1}y_{\ell}^p)^{1-p}$ as the dependent
  variable.  This worked well for establishing the projected Gamma density, and via proportionality,
  inference on the projected Gamma parameters was not affected by the choice of the summary variable.

However, when we look to establish density estimation routines for purposes of anomaly detection,
  where we do take into account the full calculated density, the choice of summary variable does affect
  the calculated density. As such, we need to consider methods that take this into account.

\makenote{needs serious rewrite}
\begin{enumerate}
  \item Density Estimation via application of KNN
  \item Explicit probability of dependent variable; projected gamma conditional on dependent variable = max
\end{enumerate}

\subsection{Local Density Estimation on the hypersphere via $k$-Nearest Neighbors}

We establish a local posterior predictive density under the $k$NN estimator by assuming a locally
  uniform density within a $d-1$ dimensional ball, centered on $\bm{V}_i$, with a radius specified by the
  negative definite kernel metric defined in \cite{trubey:pg}, to the $k$th nearest neighbor in a sample
  from the posterior predictive distribution.  The reciprocal of the volume of such a sphere,
  multiplied by $k$, the number of posterior predictive replicates inside the sphere,
  provides a rough estimate of local density around $\bm{V}_i$. Low values indicate low density.
  To make a score that interprets as larger indicates more anomalous, we take the reciprocal:
\begin{equation}
  \label{eqn:ad_knn}
    S_{i,(k)}^{\text{kNN}} = \frac{N}{k}
      \frac{\pi^{\frac{d-1}{2}}D_{k}^{d-1}(V_i)}{\Gamma\left(\frac{d-1}{2} - 1\right)}
\end{equation}
  This score can be used as specified to declare anomalies in the angular sense.  We can also
  consider the magnitude, or radial component.  Recall that $r_i \sim \text{Pareto}(1)$, then
  $P(R > r_i) = r_i^{-1}$.  A larger radial component \emph{might} be thought of as more anomalous,
  so following similar logic, to establish a score, we take $r_i$ directly.
  Multiplying the above $k$NN score with this Pareto score provides a combined score:
\begin{equation}
  \label{eqn:ad_knn_p}
  S_{i,(k)}^{\text{kNN-P}} = S_{i,(k)}^{\text{kNN}}\times r_i.
\end{equation}

Combined scores can be further weighted, to prefer the angular component or radial component score.
Let $\kappa \in [-1,1]$.  Then
\begin{equation}
  S_{i,(k),\kappa}^{\text{kNN-P}} = \left(S_{i,(k)}^{\text{kNN}}\right)^{1 + \kappa}r_i^{1-\kappa}
\end{equation}
where $\kappa = 0$ indicates no preferential weighting.

\subsection{Posterior Predictive Density via Conditional Projected Gamma}

\makenote{spitballing.  ignore.}
Recall the projected Gamma density from~\cite{trubey:pg}.
  \[
    \text{\makenote{PG equation}}
  \]
Note that the Jacobian is dependent upon $y_d$, and if $p = \infty$, the transformation becomes
  non-differentiable.  The Jacobian becomes dependent upon $x_d = \max_ix_i$.  If we pre-condition
  on $x_d = \max_ix_i$, then the determinant of the Jacobian approaches $x_d^{-1} = 1$.

\subsubsection{Contribution to posterior predictive loss}
Another way we can think of anomalies is as observations that are difficult to fit--that is, difficult
    for a fitted model to predict.  We assume that the greater the difficulty encountered in fitting a
    given observation, the more anomalous it is.  The posterior predictive loss criterion in
    Equation~\ref{eq:ppl} provides one such metric.  The first term measures the posterior predictive
    variance--a larger variance corresponds to less precision in the prediction, or alternatively, a lower
    density in the predictive distribution.  The second term measures squared bias--the squared distance
    between the average predicted value, and the observation.  Both of these terms provide a metric of
    how \emph{difficult} a particular observation is to fit.

\subsection{Kernel Density Estimation}

Gaussian kernel density estimation using kernel metric $g(\cdot,\cdot)$ defined in \cite{trubey:pg}.
\begin{equation}
    f(\bm{x}\mid \bm{X}) \approx \frac{1}{N}\sum_{i = 1}^Ne^{-\frac{1}{2}\left(\frac{g(\bm{x},\bm{X}_i^{\prime})}{h}\right)^2}
\end{equation}

blah blah blah.

\section{Mixed Model Kernel Metrics}https://www.overleaf.com/project/61aa7ebcfc1c07f984e2abf8
For strictly angular data, the kernel metric detailed in~\cite{trubey:pg} can serve as a 
  performant analogue for \emph{distance} measurement in these anomaly detection metrics.  However, 
  when we include categorical data to the analysis, this may require some adjustment of the kernel 
  metric.  We elaborate on a few possibilities.


In the first case, we can treat the categorical variables as merely sets of additional faces.  Data then lie 
    along the intersection between their \emph{angular} face, and their \emph{categorical} face.  
\begin{equation}
    D\left[(\bm{y},\bm{c}), (\bm{y}^{\prime}, \bm{c}^{\prime})\right] = g\left((\bm{y},\bm{c}), (\bm{y}^{\prime}, \bm{c}^{\prime})\right)
\end{equation}
\makenote{This actually requires some additional work/thought.  We would have to transverse  $\sum_i(c_i - c_{i}^{\prime})^2$ faces at a minimum. I'm still trying to imagine how that works--as each point would lie on the intersection of some number of faces.}

In the second case, we can regard categorical variables as canonically separate from the angular variables. 
    A kernel metric in this case could be gathered as the linear combination of two kernels--the angular, and one relating to the categorical.
    
\begin{equation*}
    D\left[(\bm{y},\bm{c}), (\bm{y}^{\prime}, \bm{c}^{\prime})\right] = g(\bm{y},\bm{y}^{\prime}) + \frac{k}{1 + k}\lVert \bm{c} - \bm{c}^{\prime}\rVert_2
\end{equation*}
    
In the third case, rather than evaluating distance to $\bm{c}_i$, we can also evaluate distance to $\bm{\pi}_i$,
    the latent categorical probability of membership.  This is likely to result in a smoother decision curve.
\begin{equation*}
    D\left[(\bm{y},\bm{c}), (\bm{y}^{\prime}, \bm{c}^{\prime})\right] = g(\bm{y},\bm{y}^{\prime}) + \frac{k}{1 + k}\lVert \bm{c} - \bm{\pi}^{\prime}\rVert_2
\end{equation*}

\[
    D\left[(\bm{y},\bm{c}), (\bm{y}^{\prime}, \bm{c}^{\prime})\right] = g(\bm{y},\bm{y}^{\prime}) + \frac{k}{1 + k}\lVert \text{E}[\bm{\pi}] - \bm{\pi}^{\prime}\rVert_2
\]

\[
    D\left[(\bm{y},\bm{c}), (\bm{y}^{\prime}, \bm{c}^{\prime})\right] = g(\bm{y},\bm{y}^{\prime}) + \frac{k}{1 + k}\lVert \text{E}[\bm{\rho}]_1 - \bm{\pi}^{\prime}\rVert_2
\]




\begin{equation*}
    D\left[(\bm{y},\bm{c}), (\bm{y}^{\prime}, \bm{c}^{\prime})\right] = g(\bm{y},\bm{y}^{\prime}) + \frac{k}{1 + k}\text{E}\left[\lVert \bm{\pi} - \bm{\pi}^{\prime}\rVert_2\right]
\end{equation*}

\[
    \text{E}\lVert (R\bm{y},\bm{\rho}), (R^\prime\bm{Y}^\prime,\bm{\rho}^\prime)\rVert_2
\]


And in the fourth case, we can project $\pi$ from $\mathbb{S}_1$ to $\mathbb{S}_{\infty}$, allowing use of
    the previously developed kernel.
\begin{equation*}
    D\left[(\bm{y},\bm{\pi}), (\bm{y}^{\prime}, \bm{\pi}^{\prime})\right] = \text{E}\left[ g\left((\bm{y},\bm{\pi}_{\infty}), (\bm{y}^{\prime},\bm{\pi}_{\infty}^{\prime})\right)\right]
\end{equation*}



The addition of categorical data to the analysis may require some adju







% EOF
