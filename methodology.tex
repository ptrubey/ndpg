
\section{Novelty Detection Methods}

Building on the notion that anomalies occur in areas of low density, we consider methods of
  estimating the statistical density of observed data.  As we saw in~\cite{trubey:pg}, density
  estimation on the surface of ${\mathbb S}_{\infty}^{d-1}$ is not an easy problem, so for purposes
  of establishing inference on the dependence structure of the data we opted to project from
  ${\mathbb S}_{\infty}^{d-1}$ onto ${\mathbb S}_p^{d-1}$ for some large but finite $p$.  This makes
  density estimation \emph{possible}, but for purposes of anomaly detection, the projection onto
  ${\mathbb S}_p^{d-1}$ is strongly dependent on \emph{which} dimension was used--that is, which
  dimension became a function of the others, hereafter referred to as the \emph{dependent} variable.
  In~\cite{trubey:pg}, we declared $y_d = (1 - \sum_{\ell = 1}^{d-1}y_{\ell}^p)^{1-p}$ as the dependent
  variable.  This worked well for establishing the projected Gamma density, and via proportionality,
  inference on the projected Gamma parameters was not affected by the choice of the summary variable.

However, when we look to establish density estimation routines for purposes of anomaly detection,
  where we do take into account the full calculated density, the choice of summary variable does affect
  the calculated density. As such, we need to consider methods that take this into account.

\makenote{needs serious rewrite}
\begin{enumerate}
  \item Density Estimation via application of KNN
  \item Explicit probability of dependent variable; projected gamma conditional on dependent variable = max
\end{enumerate}

\subsection{Local Density Estimation on the hypersphere via $k$-Nearest Neighbors}

We establish a local posterior predictive density under the $k$NN estimator by assuming a locally
  uniform density within a $d-1$ dimensional ball, centered on $\bm{V}_i$, with a radius specified by the
  negative definite kernel metric defined in \cite{trubey:pg}, to the $k$th nearest neighbor in a sample
  from the posterior predictive distribution.  The reciprocal of the volume of such a sphere,
  multiplied by $k$, the number of posterior predictive replicates inside the sphere,
  provides a rough estimate of local density around $\bm{V}_i$. Low values indicate low density.
  To make a score that interprets as larger indicates more anomalous, we take the reciprocal:
\begin{equation}
  \label{eqn:ad_knn}
    S_{i,(k)}^{\text{kNN}} = \frac{N}{k}
      \frac{\pi^{\frac{d-1}{2}}D_{k}^{d-1}(V_i)}{\Gamma\left(\frac{d-1}{2} - 1\right)}
\end{equation}
  This score can be used as specified to declare anomalies in the angular sense.  We can also
  consider the magnitude, or radial component.  Recall that $r_i \sim \text{Pareto}(1)$, then
  $P(R > r_i) = r_i^{-1}$.  A larger radial component \emph{might} be thought of as more anomalous,
  so following similar logic, to establish a score, we take $r_i$ directly.
  Multiplying the above $k$NN score with this Pareto score provides a combined score:
\begin{equation}
  \label{eqn:ad_knn_p}
  S_{i,(k)}^{\text{kNN-P}} = S_{i,(k)}^{\text{kNN}}\times r_i.
\end{equation}

Combined scores can be further weighted, to prefer the angular component or radial component score.
Let $\kappa \in [-1,1]$.  Then
\begin{equation}
  S_{i,(k),\kappa}^{\text{kNN-P}} = \left(S_{i,(k)}^{\text{kNN}}\right)^{1 + \kappa}r_i^{1-\kappa}
\end{equation}
where $\kappa = 0$ indicates no preferential weighting.

\subsection{Posterior Predictive Density via Conditional Projected Gamma}

\makenote{spitballing.  ignore.}
Recall the projected Gamma density from~\cite{trubey:pg}.
  \[
    \text{\makenote{PG equation}}
  \]
Note that the Jacobian is dependent upon $y_d$, and if $p = \infty$, the transformation becomes
  non-differentiable.  The Jacobian becomes dependent upon $x_d = \max_ix_i$.  If we pre-condition
  on $x_d = \max_ix_i$, then the determinant of the Jacobian approaches $x_d^{-1} = 1$.

\subsubsection{Contribution to posterior predictive loss}
Another way we can think of anomalies is as observations that are difficult to fit--that is, difficult
    for a fitted model to predict.  We assume that the greater the difficulty encountered in fitting a
    given observation, the more anomalous it is.  The posterior predictive loss criterion in
    Equation~\ref{eq:ppl} provides one such metric.  The first term measures the posterior predictive
    variance--a larger variance corresponds to less precision in the prediction, or alternatively, a lower
    density in the predictive distribution.  The second term measures squared bias--the squared distance
    between the average predicted value, and the observation.  Both of these terms provide a metric of
    how \emph{difficult} a particular observation is to fit.




% EOF
