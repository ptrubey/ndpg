\section{Conclusion\label{sec:conclusion}}
In this paper, we established a method of \emph{scoring} observations as 
    anomalous based on their posterior-predictive angular density, using the 
    result from multivariate extreme value theory that---assuming the existence 
    of a limiting behavior---given observations are in excess of a high 
    threshold, after transformation their angular distribution on 
    $\mathbb{S}_{\infty}^{d-1}$ is independent of the radial distribution on 
    $\mathbb{R}_+$.  In the anomaly detection setting, this independence allows
    us to separate anomaly scores into an angular and radial component, and
    treat them separately.  To establish an angular anomaly score, a Bayesian 
    non-parametric model is established on the angular data projected onto 
    $\mathbb{S}_p^{d-1}$, and as a true density on $\mathbb{S}_{\infty}^{d-1}$ 
    is not available, anomaly scores are declared using a non-parametric 
    estimator to that angular density built on a sample from the posterior 
    predictive distribution of the fitted model.  The non-parametric estimators 
    we used were $k$-nearest neighbors, and kernel density estimation.

We then expanded the model to handle categorical data, recognizing that in the 
    real world data does not always fit our assumption of the existence of a 
    limiting behavior.  We did this by establishing a Bayesian non-parametric 
    categorical data model, then tying it in with the previously declared 
    \emph{angular} model, rendering a \emph{mixed} model possible.  We explored 
    various methods of establishing an anomaly score based on the categorical 
    data, analogous to the scores established for the angular data.  For the 
    scores, rather than establishing them based on the categorical data 
    explicitly, we established them based on their latent class probability 
    vector, as calculated by the model conditional on the observation. Samples 
    of these latent probability vectors were either averaged then projected onto
    $\mathbb{S}_{\infty}^{d-1}$ as in the case of \emph{hknn} and \emph{hkde}, 
    or directly projected onto same as in the case of \emph{lhkde}, or 
    projected onto $\prod_{m = 1}^M\mathbb{S}_1^{d_{m}-1}$ as in the case 
    of \emph{lskde}.  There, pairwise estimates of distance were calculated to 
    a sample from the posterior predictive distribution, and used to calculate 
    the posterior predictive density for observation $i$. We applied the 
    categorical scores to four datasets, three of which were transformed to be 
    categorical from mixed data.  In this analysis, we observed that 
    \emph{lskde} performed reliably well.

We incorporated the categorical data model with the real data model to create a 
    mixed-data model, and adjusted the categorical scores to the mixed-data 
    regime.  We subjected six datasets to our peaks-over-threshold analysis and 
    observed that \emph{lmkde} performed reliably well, better than canonical 
    methods most of the time, but was itself outperformed in some cases by other 
    methods that project the latent probability vector along with the angular 
    vector into a unified space.

We observed that the data thresholding process may not always be applicable, so 
    finally we applied the mixed model to data with its angular component 
    transformed via the standard Pareto rank ordering transformation.  We
    acknowledge that the application of the projected gamma model to this
    transformation ignores the existence of the moment constraint implied by
    construction. In this setting, we observed the latent models (\emph{lmkde} 
    and \emph{lskde}) performed reliably well, exceeding or achieving parity
    with canonical methods in 5 of 6 tested cases.

% EOF