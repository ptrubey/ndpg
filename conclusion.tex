\section{Conclusion\label{sec:conclusion}}
In this paper, we have proposed a method of \emph{scoring} observations as 
    anomalous based on their posterior-predictive angular density, using the 
    result from multivariate extreme value theory that---assuming the existence 
    of a limiting behavior---given observations are in excess of a high 
    threshold, after transformation their angular distribution on 
    $\mathbb{S}_{\infty}^{d-1}$ is independent of the radial distribution on 
    $\mathbb{R}_+$.  In the anomaly detection setting, this independence allows
    us to separate anomaly scores into an angular and radial component, and
    treat them separately.  To define an angular anomaly score, a Bayesian 
    non-parametric model is developed on the angular data projected onto 
    $\mathbb{S}_p^{d-1}$, and as a true density on $\mathbb{S}_{\infty}^{d-1}$ 
    is not available, anomaly scores are obtained using a non-parametric 
    estimator to that angular density built on a sample from the posterior 
    predictive distribution of the fitted model.  The non-parametric estimators 
    we used were $k$-nearest neighbors, and kernel density estimation.

We then expanded the model to handle categorical data, recognizing that in the 
    real world data does not always fit our assumption of the existence of a 
    limiting behavior.  We did this by developing a Bayesian non-parametric 
    categorical data model that provides a general approach for the exploration
    of the distribution of multivariate data. This was then tied in with the
    previously defined angular model, providing an approach to mixed data 
    modelling.  We explored various methods of defining an anomaly score based 
    on the categorical data, analogous to the scores considered for the angular 
    data making use of of latent class probability vectors. We applied the 
    categorical scores to four datasets, three of which were transformed to be 
    categorical from mixed data.  In this analysis, we observed that 
    \emph{lskde} performed reliably well.

In addition, the analysis of six datasets performed with the mixed model
    indicated that \emph{lmkde} performed reliably well, better than canonical 
    methods most of the time, but was itself outperformed in some cases by other 
    methods that project the latent probability vector along with the angular 
    vector into a unified space.  Finally, As the data thresholding process 
    may not always be applicable, we applied the mixed model to data with its 
    angular component transformed via the standard Pareto rank ordering 
    transformation. In this setting, we observed that the latent 
    models---\emph{lmkde} and \emph{lskde}---performed reliably well, as well 
    or better than canonical methods in five of six tested cases.

In this paper, we have presented a highly flexible model--based method for
    anomaly detection that scales to moderately large dimensions and sample
    sizes.  However, as seen in Table~\ref{tab:data}, even for the dimensions
    and sample sizes presented, model fitting can take several hours.  Scaling
    this model beyond some thousands of observations or tens of columns
    will require a paradigm shift in \emph{how} the model is fit.  For this 
    reason, we are investigating faster means of model fitting, including
    a variational approach.

% EOF