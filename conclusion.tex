\section{Conclusion}
In this paper, we established a method of \emph{scoring} observations as anomalous based
    on their posterior-predictive angular density, using the result from multivariate extreme value
    theory that---assuming the existence of a limiting behavior---given observations are in excess of 
    a high threshold, then their angular distribution on $\mathbb{S}_{\infty}^{d-1}$ and 
    radial distribution on $\mathbb{R}_+$ are independent. A Bayesian non-parametric model 
    is established on the angular data projected onto $\mathbb{S}_p^{d-1}$, and as a 
    true density on $\mathbb{S}_{\infty}^{d-1}$ is not available, 
    anomaly scores are declared using a non-parametric estimator to that angular density 
    built on a sample from the posterior predictive distribution of the fitted model.  The
    non-parametric estimators we used were $k$-nearest neighbors, and kernel density estimation.

We then expanded the model to handle categorical data, recognizing that in the real world data
    does not always fit our assumption of the existence of a limiting behavior.  We did this
    by establishing a Bayesian non-parametric categorical data model, then tying it in with the
    previously declared \emph{angular} model, rendering a \emph{mixed} model possible.  
    We explored various methods of establishing an anomaly score based on the categorical data,
    analogous to the scores established for the angular data.
    For the scores, rather than establishing them based on the categorical data explicitly, we
    established them based on their latent class probability vector, as calculated by the model
    conditional on the observation. Samples of these latent probability vectors were either 
    averaged then projected onto $\mathbb{S}_{\infty}^{d-1}$ as in the case of \emph{hKNN} and 
    \emph{hKDE}, or directly projected onto same as in the case of \emph{lhKDE}, or projected onto
    $\prod_{m = 1}^M\mathbb{S}_1^{d_{W_m}-1}$ as in the case of \emph{lsKDE}.  There, pairwise 
    estimates of distance were calculated to a sample from the posterior predictive distribution, 
    and used to calculate the posterior predictive density for observation $i$. We applied the 
    categorical scores to four datasets, three of which were transformed to be categorical
    from mixed data.  In this analysis, we observed that \emph{lsKDE} performed reliably well.

We incorporated the categorical data model with the real data model to create a mixed-data model, 
    and adjusted the categorical scores to the mixed-data regime.  We subjected six datasets
    to our peaks-over-threshold analysis and observed that \emph{lmkde} performed reliably well,
    better than canonical methods most of the time, but was itself outperformed in some cases
    by other methods that project the latent probability vector along with the angular vector 
    into a single unified space.

We observed that the data thresholding process may not always be applicable, so we 
    
%EOFhttps://www.overleaf.com/project/61aa7ebcfc1c07f984e2abf8