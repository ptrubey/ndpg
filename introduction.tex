\section{Introduction}
\makenote{Expand upon relevance of anomaly detection in the context of extreme value theory}.

Novelty detection describes a field of methods for identifying observations as \emph{anomalous}.  That
  is to say, observations which are, in some way, \emph{different} than non-anomalous data.
  Equivalent names for this field are \emph{outlier detection}, and \emph{anomaly detection}, though
  some authors will separate the use of those terms by whether or not it is assumed that anomalous
  observations are included in the training dataset.  Crucial to this distinction is the assumption
  of a \emph{clean} training data set, containing no anomalies.

The applicability of extreme value theory to anomaly detection is predicated on the belief that
  extreme observations are more susceptible to anomalies.  \cite{goix2017} provides a discussion to
  this effect, that extreme observations exist at the border between anomalous and non-anomalous regions.

For our purpose, we do not assume the existence of labels in the training dataset, and seek an
  algorithm that can produce anomaly scores in an absence of class labels. As such, we will offer
  a brief overview of unsupervised anomaly detection methods, as well as discussion of the methods
  we are preparing here as competing models.





\makenote{roadmap of paper.  section 2) review of relevant extreme value theory; section 3) proposed models; section 4) proposed categorical adaptations and mixed models, relevant distance metrics for each; section 5) results; section 6) Conclusion}







