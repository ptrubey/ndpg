\section{Results\label{sec:results}}
As mentioned in~\ref{sec:novelty}, the goal of a novelty detection algorithm is 
    to produce a novelty \emph{score}, which ranks observations in order as to 
    the degree they are anomalous.  This creates another problem: threshold 
    selection---anomaly scores beyond what level are determined anomalous?  We 
    mentioned \cite{clifton2011} and \cite{gu2021} as examples of computing 
    thresholds theoretically, but in general, thresholds are determined 
    heuristically, using performance criteria.  In specific applications, this 
    can be extremely costly.   For instance, 

One such criteria is the receiver operating characteristics, or \emph{ROC}, 
    curve.  For  a given score threshold, one can compute the true positive 
    rate, or TPR, as the number of anomalous observations with scores above the 
    threshold, divided by the total number of anomalous observations.  The 
    false positive rate, or FPR, is similarly the number of non-anomalous 
    observations above the threshold, divided by the total number of 
    non-anomalous observations.  The ROC curve is formed as the TPR is plotted 
    on the vertical axis against the FPR on the horizontal axis for a range of 
    possible thresholds.  The curve is non-decreasing, starting at the origin 
    $(0,0)$, and ending at unity $(1,1)$.  Threshold selection might include 
    specifying an acceptable FPR, and determining the threshold that produces 
    that FPR.

In assessing model performance, we sideline the issue of threshold selection by 
    observing the whole ROC curve.  Specifically, we look for the area under 
    the ROC curve, \emph{(AuROC)}.  The better a classifier is, the closer its 
    ROC curve will approach the upper left corner, and the closer its AuROC will 
    approach 1.

\add{\bf Before we go into the results, we need to provide some information 
    about how we do the computations. I am happy to provide the details of the 
    MCMC in an appendix, or even in a supplementary material, but some idea of 
    the computations has to be given in the main body of the paper. Likewise, 
    we need to provide information on the hyperparameter choice, and how we 
    tuned the procedure, how we assessed it, and how log it took to obtain the 
    results.}\makenote{addressed all but time}

In developing our model, we employ the blocked Gibbs sampler for stick-breaking
    priors detailed in~\cite{ishwaran2001}.  We set a discount factor of $0.1$, 
    and a concentration parameter of $1.0$.  In our testing, in the neighborhood
    around these values we found the resultant number of extant clusters to be 
    relatively stable.  We use $(\mu_0 = \bm{0}_d,\Sigma_{\mu} = \bm{I}_d)$ as 
    prior parameters for $\mu$, and $(\nu = d + 50,\Psi = \nu I_d)$ as prior 
    parameters for $\Sigma$, except for the categorical components of the shape 
    vector as described in Section~\ref{sec:categorical}.  Deviations in $\mu_0$ 
    towards the negative direction bias the model towards asymptotic 
    independence, which in our testing resulted in lower model fidelity.
    To update the cluster shape vectors, we employ a joint proposal step in 
    log-space using a multivariate normal proposal, where the proposal 
    covariance is informed with an adaptive Metropolis 
    algorithm.\makenote{needs citation}.  To hasten updates to the shape
    parameters, and speed convergence of the model, we employ a parallel
    tempering algorithm where parallel MCMC chains are sampled at an ascending
    temperature ladder.  \emph{Hotter} chains have flatter posteriors, and thus 
    more readily move around the parameter space.  Chain states are randomly 
    exchanged via a Metropolis step
    with probability \makenote{add state exchange probability equation}.
    For each example dataset, the sampler was ran for $50,000$ iterations,
    discarding the first 40,000 as burn-in.  The resulting chain was thinned,
    keeping only every 10th iteration.

We tested our scores against several labelled datasets.  These included some 
    strictly categorical data, as well as categorical transformations of some 
    mixed datasets.  For mixed datasets, we tested both the thresholding and 
    rank-transformation regimes. Table~\ref{tab:data} includes an exploratory 
    analysis of our datasets, both raw and under the various testing regimes.

\begin{table}[t]
    \centering
    \caption{Characteristics of datasets used in analysis.  $N$ refers to 
    number of observations, $A$ number of anomalies, and $P$ the prevalence of 
    anomalies in the analysis set.  Peaks-over-threshold analysis uses the 
    \emph{over} set, while rank transformation and categorical analysis use 
    the \emph{sub} set.\label{tab:data}}
    \bigskip
    % latex table generated in R 4.2.2 by xtable 1.8-4 package
% Mon Jan 30 15:17:19 2023
\begin{tabular}{l|rrr|rrrr|rrr}
  \toprule
Dataset & N\_raw & A\_raw & P\_raw & Quantile & N\_over & A\_over & P\_over & N\_sub & A\_sub & P\_sub \\ 
  \midrule
annthyroid & 3600 &  270 & 0.075 & 0.850 &  715 &  150 & 0.210 & 1200 &  105 & 0.087 \\ 
  cardio & 1831 &  176 & 0.096 & 0.850 &  715 &  152 & 0.213 & 1831 &  176 & 0.096 \\ 
  cover & 19070 &  194 & 0.010 & 0.980 & 5504 &  194 & 0.035 & 1907 &   20 & 0.010 \\ 
  mammography & 11183 &  260 & 0.023 & 0.950 & 2390 &  227 & 0.095 & 1864 &   42 & 0.023 \\ 
  pima &  768 &  268 & 0.349 & 0.900 &  205 &  106 & 0.517 &  768 &  268 & 0.349 \\ 
  yeast & 1484 &   90 & 0.061 & 0.900 &  343 &   35 & 0.102 & 1484 &   90 & 0.061 \\ 
  solarflare & 1389 &   12 & 0.009 &  &  &  &  & 1389 &   12 & 0.009 \\ 
   \bottomrule
\end{tabular}

    % \makenote{Here lies a table of characteristics of datasets used in analysis.  
    % Imagine first column dataset name, second column number of obsv. original, third 
    % threshold; 4th prevalence of anomalies/above threshold, fourth number of obsv. 
    % after thresholding (include whether subsampling was used), 5th number of obsv. 
    % included in rank (include whether subsampling was used); 6th number of obsv. 
    % included in categorical (include whether subsampling was used).  that's too 
    % many columns.  need to work on this.}
\end{table}

We compared the four developed scores against three canonical novelty detection 
    algorithms, including isolation forest \emph{(iso)} \cite{liu2000}, local 
    outlier factor (lof) \cite{breunig2000}, and one-class SVM (svm) 
    \cite{chang2011}.  Each dataset was subject to 5-fold cross-validation, and 
    out-of-sample performance scores were averaged to produce the resulting 
    performance tables seen in this section.

\subsection{Categorical anomalies}
The categorical transformation of \emph{cover}, \emph{pima}, and \emph{yeast}
    discretized the real-valued and ordinal variables in those datasets.  For 
    \emph{cover} in particular, it seems this transformation lost a significant 
    amount of data.  From Table~\ref{tab:data}, it seems a large portion of 
    data regarding anomalies is contained within the radial component, so a 
    categorical transformation loses that information.  Likely for this reason, 
    none of the methods offer exceptional performance on this dataset.
\begin{table}[ht]
    \centering
    \caption{Area under the \emph{ROC} curve for various anomaly detection 
        schemes, on \emph{strictly categorical} datasets.  Reported here is 
        arithmetic mean of out-of-sample performance for 5-fold cross-validation.  
        Values closer to 1 are preferred.\label{tab:perfcat}}
    \bigskip
    % latex table generated in R 4.2.2 by xtable 1.8-4 package
% Mon Jan 30 15:17:28 2023
\begin{tabular}{l|rrr|rrrr}
  \toprule
dataset & iso & lof & svm & hknn & hkde & lhkde & lskde \\ 
  \midrule
cover & 0.384 & 0.515 & 0.424 & $\bm{0.586}$ & 0.523 & 0.558 & 0.450 \\ 
  pima & 0.620 & 0.570 & 0.614 & 0.457 & 0.579 & 0.659 & $\bm{0.694}$ \\ 
  solarflare & $\bm{0.893}$ & 0.402 & 0.887 & 0.435 & 0.632 & 0.768 & 0.875 \\ 
  yeast & 0.620 & 0.580 & 0.622 & 0.406 & $\bm{0.708}$ & 0.650 & 0.702 \\ 
   \bottomrule
\end{tabular}

\end{table}
The dataset \emph{solarflare} was also unique in our analysis, being the only 
    truly categorical dataset used.  Our algorithm \emph{lsKDE} very slightly 
    trailed the performance of \emph{one-class SVM}, the best performing 
    algorithm on this dataset.  On both \emph{pima} and \emph{yeast}, 
    latent-simplex KDE performed significantly better than any of the canonical 
    methods.  On this analysis, \emph{hKDE} and \emph{hKNN} both performed 
    poorly.  It seems the projection of the categorical probability vectors into 
    a unified sphere induces some loss of information.

\subsection{Peaks-over-Threshold anomalies}
We subjected six datasets to multivariate thresholding, only keeping 
    observations that exceeded the threshold in at least one dimension.  For 
    most datasets, this threshold was set at $\hat{F}_{\ell}^{-1}(0.95)$ for 
    each dimension.  For \emph{cover}, we sub-sampled the data to produce a 
    more manageable sized dataset.  For variables on these datasets that did 
    not exhibit properties that would allow for a peak-over-threshold model to 
    apply, these variables were instead converted to discrete values with two 
    or three categories.  We built the mixed data model, and evaluated 
    performance of the mixed scores, compared against the canonical methods.  
    Of particular interest here is the \emph{annthyroid} dataset, for which all 
    of our scores performed comparably, and significantly better than the 
    canonical scores.  Of the other tested datasets, on \emph{cardio}, 
    \emph{lmKDE} approached the performance of \emph{isolation forest} and 
    \emph{one-class SVM}, but all other methods performed worse.  For the 
    datasets \emph{cover} and \emph{mammography}, \emph{hKNN}, \emph{lhKDE}, 
    and \emph{lmKDE} performed comparably, and each significantly better than 
    any of the canonical methods.  We see that \emph{lmKDE}, being the inheritor
    of the latent simplex KDE score, performs reasonably well reliably among 
    datasets thus far in the peaks-over-threshold setting, but is outperformed 
    by other metrics on each dataset. We may see some effect of the loss of 
    information relating to the dependence structure between $\bm{w}$ and 
    $\bm{v}$ on the derived performance.  On that note, \emph{lhKDE} performed 
    comparably to \emph{lmKDE} on \emph{annthyroid}, \emph{cover}, \emph{pima}, 
    and \emph{yeast}, but slightly exceeded its performance on 
    \emph{mammography}.  We saw in the categorical datasets, \emph{lsKDE} 
    performed generally well, so the projection onto a unified sphere may 
    induce loss of information.  In that regard, it may be the case that 
    preserving information about the dependence structure between $\bm{v}$ and 
    $\bm{w}$ had a greater effect than a greater effect than preserving 
    information within $\bm{w}$ specifically.
    
\begin{table}[ht]
    \centering
    \caption{Area under the \emph{ROC} curve for various anomaly detection 
        schemes, on \emph{mixed} data where the real component has undergone 
        the \emph{threshold} standard Pareto transformation.   Reported here is 
        arithmetic mean of out-of-sample performance for 5-fold 
        cross-validation.  Values closer to 1 are preferred.
        \label{tab:perfreal}}
    \bigskip
    % latex table generated in R 4.2.2 by xtable 1.8-4 package
% Mon Jan 30 15:17:28 2023
\begin{tabular}{l|rrr|rrrr}
  \toprule
dataset & iso & lof & svm & hknn & hkde & lhkde & lmkde \\ 
  \midrule
annthyroid & 0.458 & 0.512 & 0.640 & 0.691 & 0.692 & 0.698 & 0.689 \\ 
  cardio & 0.849 & 0.610 & 0.836 & 0.590 & 0.812 & 0.804 & 0.823 \\ 
  cover & 0.606 & 0.512 & 0.684 & 0.832 & 0.698 & 0.719 & 0.714 \\ 
  mammography & 0.594 & 0.616 & 0.725 & 0.675 & 0.750 & 0.757 & 0.725 \\ 
  pima & 0.530 & 0.565 & 0.511 & 0.525 & 0.525 & 0.524 & 0.522 \\ 
  yeast & 0.427 & 0.579 & 0.560 & 0.639 & 0.522 & 0.540 & 0.542 \\ 
   \bottomrule
\end{tabular}

\end{table}

As to the poor performance of every method on \emph{pima} and \emph{yeast}, 
    these reported \emph{AuROC} values are conditional on the data exceeding 
    the multivariate threshold used in building the model.  As we see in 
    Table~\ref{tab:data}, these datasets do not meet the assumption that 
    anomalies are concentrated in the tails.  Scores depending on $r_i$, the
    radius component of $\bm{z}_i$, or \emph{magnitude} of the extremal 
    observation, are going to perform poorly relative to metrics that do not 
    make that assumption.

\subsection{Rank Transformation anomalies}
We subjected the same six datasets used in the peak-over-threshold model to 
    rank transformation on the real and ordinal variables. We then built the 
    mixed model including radius described in Section~\ref{subsec:rank} on the 
    transformed datasets.  For extremely large datasets, we sub-sampled the 
    data down to a manageable size.
    \makenote{need to rewrite sentence}  
    Note that rank transformation preserves the entire dataset, so we should 
    not consider the values in Table~\ref{tab:perfrank} to be comparable to 
    the values in Table~\ref{tab:perfreal}.
    
\begin{table}[ht]
    \centering
    \caption{Area under the \emph{ROC} curve for various anomaly detection 
        schemes, on \emph{mixed} data where the real component has undergone 
        the \emph{rank} standard Pareto transformation. Reported here is 
        arithmetic mean of out-of-sample performance for 5-fold 
        cross-validation.  Values closer to 1 are preferred.
        \label{tab:perfrank}}
    \bigskip
    % latex table generated in R 4.2.2 by xtable 1.8-4 package
% Mon Jan 30 15:17:28 2023
\begin{tabular}{l|rrr|rrrr}
  \toprule
dataset & iso & lof & svm & hknn & hkde & lhkde & lmkde \\ 
  \midrule
annthyroid & 0.519 & 0.561 & 0.796 & 0.714 & 0.817 & $\bm{0.823}$ & 0.822 \\ 
  cardio & $\bm{0.887}$ & 0.588 & 0.634 & 0.648 & 0.847 & 0.848 & 0.883 \\ 
  cover & 0.898 & 0.680 & 0.931 & 0.833 & 0.960 & 0.960 & $\bm{0.960}$ \\ 
  mammography & 0.896 & 0.806 & $\bm{0.940}$ & 0.700 & 0.928 & 0.930 & 0.845 \\ 
  pima & 0.679 & 0.653 & 0.712 & 0.654 & 0.712 & 0.707 & $\bm{0.714}$ \\ 
  yeast & $\bm{0.675}$ & 0.527 & 0.632 & 0.566 & 0.601 & 0.593 & 0.599 \\ 
   \bottomrule
\end{tabular}

\end{table}

Here \emph{lmKDE} performs better than each of the canonical methods in four of 
    six datasets, performing slightly worse than \emph{one-class SVM} on 
    \emph{mammography}, and significantly worse than \emph{isolation forest} on 
    \emph{yeast}.  As we have stated before, \emph{yeast} and \emph{pima} are 
    datasets that do not quite meet our assumptions as to how anomalies are 
    distributed, but our methods still make a strong showing on \emph{pima}.
    

% EOF